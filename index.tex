% Created 2022-04-18 Mon 21:00
% Intended LaTeX compiler: pdflatex
\documentclass[11pt]{article}
\usepackage[utf8]{inputenc}
\usepackage[T1]{fontenc}
\usepackage{graphicx}
\usepackage{grffile}
\usepackage{longtable}
\usepackage{wrapfig}
\usepackage{rotating}
\usepackage[normalem]{ulem}
\usepackage{amsmath}
\usepackage{textcomp}
\usepackage{amssymb}
\usepackage{capt-of}
\usepackage{hyperref}
\author{Ryo Haruyama}
\date{\today}
\title{Ryo Haruyama}
\hypersetup{
 pdfauthor={Ryo Haruyama},
 pdftitle={Ryo Haruyama},
 pdfkeywords={},
 pdfsubject={},
 pdfcreator={Emacs 27.2 (Org mode 9.4.4)}, 
 pdflang={English}}
\begin{document}

\maketitle

\section*{Bio}
\label{sec:org8f50b3f}
Working at Fujii Nenshi \& Co. Ltd. Born and raised in 北海道沼田町.

\section*{Interests}
\label{sec:orgbb70fcd}
\begin{itemize}
\item Linear Logic
\end{itemize}

\section*{Socials}
\label{sec:orgc56523e}
\begin{itemize}
\item \href{https://github.com/rharuyama/}{GitHub}
\item \href{https://twitter.com/RyoHaruyama}{Twitter}
\item \href{https://connpass.com/user/Ryo\_Haruyama/}{connpass}
\item Clubhouse: @ryoharuyama
\item \href{https://www.amazon.co.jp/hz/wishlist/ls/3R1LX8E4SHIG6}{Wishlist}
\item \href{https://line.me/ti/p/CZo-uvtQ-p}{LINE}
\end{itemize}

\section*{Email}
\label{sec:org56856de}
\begin{itemize}
\item ryohあnagoya-u.jp
\end{itemize}

\section*{Slides}
\label{sec:orge91a688}
\begin{itemize}
\item \href{./phase-soundness.pdf}{Soundness for Linear Logic regarding Phase Semantics} (in English)
\item \href{./categorical-semantics-of-linear-logic.pdf}{Categorical Semantics of Linear Logic} (in Japanese)
\end{itemize}

\section*{Blog}
\label{sec:orge10e4d4}
\begin{itemize}
\item \href{./nara20220112.html}{奈良観光記録}
\item \href{./arduino.html}{Arduinoでエアコン自動化}
\item \href{./niseko-note.html}{ニセコエリア視察メモ}
\end{itemize}

\section*{Education}
\label{sec:org3968020}
\begin{itemize}
\item Graduate School of Informatics, Nagoya University - Master of Informatics, 2019-2022

\item Department of Natural Science Informatics at the School of Informatics and Sciences (now School of Informatics), Nagoya University - Bachelor of Informatics and Sciences, 2015-2019

\item 北海道旭川東高等学校 - 2012-2015
\end{itemize}
\end{document}
